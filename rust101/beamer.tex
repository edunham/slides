% Thanks to Kevin D. McGrath & Wade Cline for the presentation template.
\documentclass[50pt]{beamer}

% Packages.
\usepackage{amsmath}
\usepackage{amssymb}
\usepackage{amsthm}
\usepackage{graphicx}
\usepackage{hyperref}
\usepackage{pst-node}
\usepackage{pstricks}
\usepackage{url}
\usepackage{epigraph}
\usepackage{textcomp}

% Hyperlinks.
\def\name{E. Dunham}
\hypersetup{
  colorlinks = true,
  citecolor = black,
  linkcolor = black,
  urlcolor = black,
  pdfauthor = {\name},
  pdftitle = {Rust 101, Linux.Conf.Au 2017},
  pdfsubject = {Rust 101, Linux.Conf.Au 2017},
  pdfpagemode = UseNone
}

% ???
\usetheme[hideothersubsections]{Hannover}
\usecolortheme{sidebartab}
\useinnertheme{rounded}


% Presentation metadata.
\title[Rust 101]{Rust 101}
\subtitle{Deploy, Write, and Improve Rust}
\author{\name}
\date{2017-01-19}

\AtBeginSection[]
{
  \begin{frame}<beamer>{}
\tableofcontents[currentsection]
  \end{frame}
}


%\addtobeamertemplate{frametitle}{
%   \let\insertframetitle\insertsectionhead\insertsubsectionhead}{}

\begin{document}

\begin{frame}
\titlepage
\end{frame}

\section{Intro}

\begin{frame}
    Welcome!
     \begin{itemize}
        \item 100 minutes
        \item ``101" is an intro class
        \begin{itemize}
            \item Learn what Rust is \& isn't
            \item Run Rust code
            \item Meet Rust's special features
            \item Improve Rust
        \end{itemize}
        \item Not:
        \begin{itemize}
            \item Hack time for you
            \item Unsafe or advanced Rust
            \item Exhaustive Q\&A \footnote{Come to the BoF at lunch!}
        \end{itemize}
    \end{itemize}
\end{frame}

\begin{frame}

\epigraph{You'll learn the key concepts necessary for successful Rust
programming, as well as how to continue exploring the language after
LCA.}{\textit{The Abstract}}

\end{frame}

\subsection{About Me}

\begin{frame}
    About Me:
    \begin{itemize}
        \item DevOps for Mozilla Research \footnote{Infra is mostly Python}
        \item Rust Community Team member
        \item FOSS \& robotics background
        \item NOT a compiler wizard
    \end{itemize}
\end{frame}

\subsection{About You}

\begin{frame}
    Have you...
    \begin{itemize}
        \item Heard of Rust?
        \item Used Rust?
        \item Contributed to Rust?
    \end{itemize}
\end{frame}

\begin{frame}
    Have you...
    \begin{itemize}
        \item Written C, C++, Assembly, etc?
        \item Written Java, Python, Ruby, JS, etc?
        \item Written Haskell, Erlang, Ocaml, etc?
    \end{itemize}
\end{frame}

\begin{frame}
    Have you...
    \begin{itemize}
        \item Used version control?
        \item Used GitHub?
        \item Contributed to a FOSS project?
    \end{itemize}
\end{frame}

\subsection{About Rust}

\begin{frame}
    What's Rust?
    \begin{itemize}
        \item Systems language \footnote{Contrast to Go, a language for sysadmins}
        \item Safety + Performance
        \item Community... (Thriving but controversial)
    \end{itemize}
\end{frame}

\begin{frame}
    Rust's Buzzwords
    \begin{itemize}
        \item Safety, Speed, Concurrency
        \item Memory safety without garbage collection
        \item Zero-cost abstractions
        \item Hack Without Fear
    \end{itemize}
\end{frame}

\begin{frame}
    Aside: Safety \& GC
    \begin{itemize}
        \item Memory must be reused
        \item C: ``Just follow these rules perfectly, you're smart"
        \item Java, JS, etc: ``Wait a minute, I'll take care of it"
        \item Rust: ``I'll prove correctness at compile time"
    \end{itemize}
\end{frame}

\begin{frame}
    History
    \begin{itemize}
        \item Since \textasciitilde 2010
        \item 1.0 Stable in May 2015
        \item Currently \footnote{until January 26th} 1.14.0
        \item Mozilla sponsorship \& support
    \end{itemize}
\end{frame}

\begin{frame}
    Notable Projects
    \begin{itemize}
        \item \url{servo.org} Browser Engine
        \item \url{habitat.sh} Infrastructure Tooling
        \item Dropbox (internal use)
        \item \url{https://www.rust-lang.org/en-US/friends.html}
    \end{itemize}
\end{frame}

\begin{frame}
    Where is Rust a good tool?
    \begin{itemize}
        \item Speed + Safety essential
        \item LLVM-supported architecture
        \item Team $\heartsuit$ new technology
    \end{itemize}
\end{frame}

\begin{frame}
    Where might Rust be a bad tool?
    \begin{itemize}
        \item Timeframe prohibits new learning
        \item Need code reuse \footnote{Corrode can translate C to unsafe Rust}
        \item Can't handle CoC
    \end{itemize}
\end{frame}

\begin{frame}
    Questions about what's going on here?
\end{frame}

\section{Run Rust}

\subsection{Channels}

\begin{frame}
    Rust's channels
    \begin{itemize}
        \item Nightly: Trying neat ideas
        \item Beta: Release candidates
        \item Stable: Always backwards-compatible \footnote{\url{https://blog.rust-lang.org/2015/05/15/Rust-1.0.html}}
    \end{itemize}
\end{frame}

\begin{frame}
    Which channel to use?
    \begin{itemize}
        \item Stable code should run anywhere
        \item Switch to nightly for dependencies
        \item New project? Pick stable \footnote{Unless you need an
              unstable feature}
    \end{itemize}
\end{frame}

\begin{frame}
    Aside: Crater
    \begin{itemize}
        \item Compile all published libraries
        \item Diff results from stable and candidate
        \item \url{https://github.com/brson/taskcluster-crater}
    \end{itemize}
\end{frame}

\begin{frame}
    Questions about channels?
\end{frame}


\subsection{Installation Options}

\begin{frame}
    Installation
    \begin{itemize}
        \item Online: \url{play.integer32.com} or \url{play.rust-lang.org}
        \item Many rusts: \url{rustup.rs}
        \item System package manager \footnote{Or add Rust to your favorite!}
        \item Tinfoil hat: Compile from source
    \end{itemize}
\end{frame}

\begin{frame}
    Playpens:
    \begin{itemize}
        \item Source at \url{https://github.com/rust-lang/rust-playpen}
        \item Choose output, LLVM IR, or ASM
        \item Gist your progress
        \item Config alters editor settings
    \end{itemize}
\end{frame}

\begin{frame}
    Rustup:
    \begin{itemize}
        \item Docs at \url{https://github.com/rust-lang-nursery/rustup.rs}
        \item \textasciitilde /.cargo/bin
        \item \textit{rustup install nightly}
        \item \textit{rustup run nightly cargo new}
    \end{itemize}
\end{frame}

\begin{frame}
    Security:
    \begin{itemize}
        \item All releases GPG signed
        \item Key only held by Rust Core Team members
        \item \url{keybase.io/rust} identity proved by locally signing artifact
        \item \url{github.com/rust-lang/rust}
    \end{itemize}
\end{frame}

\begin{frame}
    Questions about installation?
\end{frame}


\section{Write Rust}

\begin{frame}
    IDE support
    \begin{itemize}
        \item \url{https://github.com/murarth/rusti}
        \item \url{http://www.jonathanturner.org/2017/01/rls-alpha-release.html}
    \end{itemize}
\end{frame}

\begin{frame}
    REPL equivalents
    \begin{itemize}
        \item Use the playpen
        \item playbot on IRC
        \item \url{https://github.com/murarth/rusti} worked briefly on nightly
    \end{itemize}
\end{frame}


\subsection{Errors}

\begin{frame}
    The compiler is your friend
    \begin{itemize}
        \item Rules catch things that look unsafe
        \item ``Unsafe" directive is an override
        \item Errors deserve helpful docs
        \item Click error number in playpen!
    \end{itemize}
\end{frame}

\begin{frame}
    Erroneous Errors
    \begin{itemize}
        \item Gist your code
        \item Ask on IRC \#rust-beginners
        \item File a bug
    \end{itemize}
\end{frame}

\begin{frame}
    Aside: Other helpful tools
    \begin{itemize}
        \item \url{https://github.com/nrc/rustfmt} standardizes style for you
        \item \url{https://github.com/manishearth/rust-clippy} gives helpful suggestions
    \end{itemize}
\end{frame}

\begin{frame}
    Questions about errors?
\end{frame}


\subsection{Syntax}

\begin{frame}
    Scopes:
    \begin{itemize}
        \item Everything between matched \{\}
        \item Scopes can nest
        \item \{Outer Scope \{Inner Scope\}\}
    \end{itemize}
\end{frame}

\begin{frame}[fragile]
    Functions:
    \begin{itemize}
        \item \textit{ fn myfunction \{ ... \} }
        \item \textit{ fn myfunction \(arg: type, arg: type\) \texttt{-} \textrangle{} resulttype \{ ... \} }
        \item Type signatures are like Mad Libs
        \item Function has at least name and scope
    \end{itemize}
\end{frame}

\begin{frame}[fragile]
    Macros:
    \begin{itemize}
        \item Shorthand for functions with variable number of arguments
        \item \textit{macroname!(foo, bar, baz)}
        \item \url{ doc.rust-lang.org/beta/book/macros.html }
    \end{itemize}
\end{frame}

\begin{frame}[fragile]
    A Function:

\begin{verbatim}
    fn halve(x: i32) -> i32 {
        return x / 2;
    }

    fn main(){
        println!("{}", halve(4));
    }
\end{verbatim}

\end{frame}

\begin{frame}
    Punctutation Matters:
    \begin{itemize}
        \item Expressions end with a semicolon
        \item Exception: bare expression on last line of function returns result
        \item Spaces must separate tokens
        \item Whitespace is mostly irrelevant
    \end{itemize}
\end{frame}

\begin{frame}[fragile]
    Abusing Whitespace:

\begin{verbatim}
    fn halve
    (x                :
    i32)->i32
    {
    return
    x
            /
                      2
    ; } fn main
    ()
    { println!("{}", halve
    (
    4
    ));}
\end{verbatim}

\end{frame}

\begin{frame}
    Control Flow
    \begin{itemize}
        \item Conditionals and loops are familiar
        \item Match statements combine conditionals
        \item \url{https://doc.rust-lang.org/book/if.html}
        \item \url{https://doc.rust-lang.org/book/loops.html}
        \item \url{https://doc.rust-lang.org/book/match.html}
    \end{itemize}
\end{frame}

\begin{frame}[fragile]
Matching on a variable:
\begin{verbatim}
fn main() {
    let day = 19;
    println!("January {} 2017 is:", day);
    match day {
        15 => println!("Travel to Hobart"),
        16 | 17 => println!("Miniconf Time"),
        18...20 => println!("The Conference"),
        _ => println!("not LCA at all"),
    }
}
\end{verbatim}
\end{frame}
\begin{frame}
    Questions about syntax?
\end{frame}


\subsection{Types \& Traits}

\begin{frame}
    Why Types \& Traits?
    \begin{itemize}
        \item Describe characteristics of inputs and outputs
        \item Avoid allocating unneeded memory
        \item Remind humans how code works \footnote{Even type signatures that the compiler could infer must be spelled out.}
    \end{itemize}
\end{frame}

\begin{frame}
    Built-in types
    \begin{itemize}
        \item \url{https://doc.rust-lang.org/book/primitive-types.html}
        \item Primitives, arrays, strings, tuples
    \end{itemize}
\end{frame}

\begin{frame}
    Custom types
    \begin{itemize}
        \item \url{http://rustbyexample.com/custom_types.html}
        \item Structs \& Enums
        \item Use types from your dependencies (example in a few slides)
    \end{itemize}
\end{frame}

\begin{frame}
    Traits
    \begin{itemize}
        \item \url{https://doc.rust-lang.org/book/traits.html}
        \item Traits describe a type's abilities
        \item You can tell Rust how type has trait with ``impl''
        \item Generalize function's input and output
    \end{itemize}
\end{frame}

\begin{frame}
    Questions about types \& traits?
\end{frame}



\subsection{Safety}

\begin{frame}
    Ownership:
    \begin{itemize}
        \item \textit{let myint = 42;}
        \item ``myint" is a variable binding
        \item ``myint" \textbf{owns} the value 42
        \item Every value has exactly one owner
        \item See \url{https://doc.rust-lang.org/book/ownership.html}
    \end{itemize}
\end{frame}

\begin{frame}
    Mutability:
    \begin{itemize}
        \item Owner can only change value if it's \textbf{mutable}
        \item \textit{let mut myint = 42;}
    \end{itemize}
\end{frame}

\begin{frame}[fragile]
    Changing the owner:
\begin{verbatim}

    fn main() {
        let first = 42;
        println!("{}", first);

        let second = first;
        println!("{}", second);

        // this would be an error:
        // println!("{}", first);
    }

\end{verbatim}

\end{frame}

\begin{frame}
    Borrowing
    \begin{itemize}
        \item Grant temporary access to a value
        \item 1 mutable borrow XOR unlimited immutable borrows
        \item Syntax: \&myvar
        \item \url{https://doc.rust-lang.org/book/references-and-borrowing.html}
    \end{itemize}
\end{frame}

\begin{frame}
    Lifetimes
    \begin{itemize}
        \item Remember \{scopes\}?
        \item Variables disappear when their scope ends!
        \item No borrow may outlive its value's owner.
    \end{itemize}
\end{frame}
\begin{frame}
    Questions about safety?
\end{frame}


\subsection{Using Libraries}

\begin{frame}
    Package Management
    \begin{itemize}
        \item Package manager: Cargo
        \item Libraries: Crates
        \item Package index: crates.io
        \item See ``Package Managers All The Way Down", Tasman B, 3:40pm today
        \item See \url{http://doc.crates.io/guide.html}
    \end{itemize}
\end{frame}

\begin{frame}
    Create a binary or library
    \begin{itemize}
        \item \textit{cargo new --bin}
            \begin{itemize}
                \item Binary
                \item You run main.rs
                \item cargo.lock stores state of last good build
            \end{itemize}
        \item \textit{cargo new}
            \begin{itemize}
                \item Library
                \item src/lib.rs, no main function
            \end{itemize}
    \end{itemize}
\end{frame}

\begin{frame}
    `cargo new' creates:
    \begin{itemize}
        \item Cargo.toml
        \item src/main.rs or src/lib.rs
        \item .git if absent
    \end{itemize}
\end{frame}

\begin{frame}[fragile]
    Depending on a crate
    \begin{itemize}
        \item Search crates.io
        \item Search the web, check recent blogs
        \item Check docs, license, \& project policies
        \item Add it to `dependencies' section of cargo.toml
        \begin{itemize}
            \item \textit{ name = ``0.1" }
            \item \textit{ name = \{ git = ``https://github.com/org/repo.git" rev = ``123abcd" \} }
        \end{itemize}
        \item \textit{ extern crate rand; use rand::Rng; }
    \end{itemize}
\end{frame}

\begin{frame}
    Questions about libraries?
\end{frame}

\section{Improve Rust}

\begin{frame}
    Always...
    \begin{itemize}
        \item Respect others' licenses
        \item License your own code
        \item Document \& share what you learn!
    \end{itemize}
\end{frame}


\subsection{Level up}

\begin{frame}
    Read a Book
    \begin{itemize}
        \item \url{https://doc.rust-lang.org/stable/book/}
        \item \url{https://doc.rust-lang.org/nomicon/}
        \item O'Reilly Book coming soon
    \end{itemize}
\end{frame}

\begin{frame}
    Follow the News
    \begin{itemize}
        \item \url{http://www.newrustacean.com/} podcast
        \item \url{https://soundcloud.com/posix4e/sets/rustyradio} interviews
        \item \url{https://this-week-in-rust.org/} Weekly newsletter
        \item \url{https://blog.rust-lang.org/} Official Blog
    \end{itemize}
\end{frame}

\begin{frame}
    Practice
    \begin{itemize}
        \item \url{https://github.com/carols10cents/rustlings}
        \item \url{http://rustbyexample.com/}
    \end{itemize}
\end{frame}

\begin{frame}
    Questions about learning more Rust?
\end{frame}


\subsection{Find a project}

\begin{frame}
    Join a project
    \begin{itemize}
        \item \url{https://crates.io/}, find popular crates
        \item Search GitHub ``is:issue label:easy language:rust"
    \end{itemize}
\end{frame}

\begin{frame}
    Port something
    \begin{itemize}
        \item \url{https://github.com/jameysharp/corrode}
        \item \url{linux.conf.au/schedule/presentation/51/} Friday 1:20pm
        \item \url{https://blog.rust-lang.org/2015/04/24/Rust-Once-Run-Everywhere.html}
    \end{itemize}
\end{frame}

\begin{frame}
    Questions about finding a project?
\end{frame}

\subsection{Get involved}

\begin{frame}
    File or Fix issues
    \begin{itemize}
        \item \url{github.com/rust-lang/rust}
        \item We triage regularly
        \item If in doubt, ask on IRC first
    \end{itemize}
\end{frame}

\begin{frame}
    Chat Online
    \begin{itemize}
        \item IRC: \#rust, \#rust-beginners on irc.mozilla.org
        \item \url{users.rust-lang.org} Users Forum
        \item Reddit, StackOverflow, etc.
    \end{itemize}
\end{frame}

\begin{frame}
    Find or join a meetup
    \begin{itemize}
        \item Search for your area + Rust meetup
        \item Community Team Calendar, \url{goo.gl/EJ2iRb}
        \item \#rust-community on Mozilla IRC
    \end{itemize}
\end{frame}

\begin{frame}
    Attend a conference
    \begin{itemize}
        \item \url{rustconf.com} Oregon, September
        \item \url{www.rust-belt-rust.com}, Pennsylvania, October
        \item \url{www.rustfest.eu}, Europe, September
    \end{itemize}
\end{frame}

\begin{frame}
    Questions about getting involved?
\end{frame}

\begin{frame}
    Attend the LCA Rustlang BoF!
    \begin{itemize}
        \item Right here, lunch today
        \item Start Hacking
    \end{itemize}
\end{frame}

\begin{frame}

Thank you!

\end{frame}
%
%\begin{frame}
%    \begin{itemize}
%        \item
%        \item
%        \item
%    \end{itemize}
%\end{frame}
%

\end{document}
