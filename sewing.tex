\documentclass{beamer}
%\usepackage{hyperref}
\usetheme{Goettingen}
%\beamerdefaultoverlayspecification{<+->}

%gets rid of bottom navigation bars
\setbeamertemplate{footline}[text line]{\url{http://talks.edunham.net/toorcamp2018}}

%gets rid of navigation symbols
\setbeamertemplate{navigation symbols}{}

\title{FABRICation}
\subtitle{How to sew pretty much anything}
\author{$@$qedunham}
\institute{ToorCamp 2018}
\date{June 21, 2018}
\begin{document}

\begin{frame}[fragile]
\titlepage
\end{frame}

\section{Intro}

\begin{frame}[fragile]
\frametitle{Hi!}
\begin{itemize}[<+(1)->]
\item I make stuff
\item No formal training
\item Thanks, Mum!
\end{itemize}
\end{frame}

\begin{frame}[fragile]
\frametitle{Just Curious...}
You...
\begin{itemize}[<+(1)->]
\item Just here to get indoors?
\item Want to learn to sew?
\item Know a bit, want to learn more?
\item Already expert?
\end{itemize}
\end{frame}

\begin{frame}[fragile]
\frametitle{Outline}
\tableofcontents[pausesections]
\end{frame}

\section{Materials}

\begin{frame}[fragile]
\tableofcontents[currentsection]
\end{frame}

\subsection{Thread}

\begin{frame}[fragile]
\frametitle{Materials}
\begin{itemize}[<+(1)->]
\item Natural fibers
    \begin{itemize}
        \item Silk, cotton, linen, wool
    \end{itemize}
\item Natural polymers
    \begin{itemize}
        \item Rayon, viscose, bamboo
    \end{itemize}
\item Synthetic polymers
    \begin{itemize}
        \item Polyester, nylon, acrylic, elastane/spandex
    \end{itemize}
\end{itemize}
\end{frame}

\begin{frame}[fragile]
\frametitle{Burn Test}
\url{https://www.craftsy.com/sewing/article/burn-test-for-fiber-content/}
\begin{itemize}[<+(1)->]
\item Cotton
    \begin{itemize}
        \item Smells like burning paper
        \item Keeps burning quickly
    \end{itemize}
\item Linen
    \begin{itemize}
        \item Burns like cotton
        \item Slight afterglow
    \end{itemize}
\item Rayon
    \begin{itemize}
        \item Burns faster than cotton
        \item No afterglow after burning
    \end{itemize}
\item Silk
    \begin{itemize}
        \item Curls away and smolders
        \item Doesn't stay lit
        \item Dissolves in bleach
        \item Smells like burnt hair
    \end{itemize}
\item Wool
    \begin{itemize}
        \item Burns like silk, but smellier
        \item Feels rougher to the touch
    \end{itemize}
\end{itemize}
\end{frame}


\begin{frame}[fragile]
\frametitle{Burn Test}
\url{https://www.craftsy.com/sewing/article/burn-test-for-fiber-content/}
\begin{itemize}[<+(1)->]
\item Polyester
    \begin{itemize}
        \item Melts
        \item Won't stay on fire
        \item No ash, just plastic beads
        \item Smells toxic
    \end{itemize}
\item Nylon
    \begin{itemize}
        \item Like polyester
        \item Less smell
        \item Melts more
    \end{itemize}
\item Spandex
    \begin{itemize}
        \item Shrinks away, melts slowly
        \item Leaves sticky goo
    \end{itemize}
\item Acetate
    \begin{itemize}
        \item Melts and drips
        \item Stinks
        \item Leaves plastic beads
    \end{itemize}
\end{itemize}
\end{frame}

\begin{frame}[fragile]
\frametitle{Choosing thread}
    \begin{itemize}[<+(1)->]
    \item Material
        \begin{itemize}
            \item Transparent? Stretch?
            \item Weaker than fabric!
        \end{itemize}
    \item Color
        \begin{itemize}
            \item Will it show?
        \end{itemize}
    \item Quality
        \begin{itemize}
            \item Strength
            \item Not too fuzzy
            \item No knots or lumps
        \end{itemize}
    \item Thickness
        \begin{itemize}
            \item Proportionate to fabric
        \end{itemize}
    \end{itemize}
\end{frame}

\subsection{Fabric}

\begin{frame}[fragile]
\tableofcontents[currentsubsection]
\end{frame}

\begin{frame}[fragile]
\frametitle{Fabrics made from thread}
\begin{itemize}[<+(1)->]
\item Knit
\item Woven
\item Felted
\end{itemize}
\end{frame}

\begin{frame}[fragile]
\frametitle{Other Fabrics}
\begin{itemize}[<+(1)->]
\item Leather
\item Plastic sheets
\item Tyvek house paper
\item Fail when perforated too much
\item Can be slightly stretchy
\item Often don't drape like cloth
\item Pin holes never go away
\end{itemize}
\end{frame}

\begin{frame}[fragile]
\frametitle{Felts}
\begin{itemize}[<+(1)->]
\item Like dreadlocks
\item Usually no stretch
\item Fail at points of stress or wear
\end{itemize}
\end{frame}

\begin{frame}[fragile]
\frametitle{Knits}
\begin{itemize}[<+(1)->]
\item Like your t-shirt or socks
\item Stretchy, even if non-stretch thread
\item Fails by unravelling ("runs")
\item Broken thread can yield ladder or small hole
\end{itemize}
\end{frame}

\begin{frame}[fragile]
\frametitle{Fake Furs}
\begin{itemize}[<+(1)->]
\item Knit backing with pile
\item Grain = direction the hair points
\item Align pattern so hairs point the right way
\item Cut only backing, not hairs
\item Melt or sew edge to reduce fraying and shedding
\item Comb or brush hairs out of seam after sewing
\end{itemize}
\end{frame}

\begin{frame}[fragile]
\frametitle{Wovens}
\begin{itemize}[<+(1)->]
\item Like your jeans
\item Only stretchy if the fibers stretch
\item Edges fail by fraying
\item Many threads must break to make a hole
\item Many different weaves (dress shirt vs denim vs brocade)
\end{itemize}
\end{frame}

\begin{frame}[fragile]
\frametitle{Fight Fraying}
\begin{itemize}[<+(1)->]
\item Chemically: Fray-check or glue
\item Mechanically: Pinking shears
\item Strategically: Leave seam allowance
\item Choose enclosed seams
\item Sew or serge edges
\end{itemize}
\end{frame}

\begin{frame}[fragile]
\frametitle{Other terms}
\begin{itemize}[<+(1)->]
\item Grain: Warp and weft of the loom
\item Bias: 45 degrees from grain
\item Right side: Ends up visible in finished work
\item Wrong side: Not the right side
\end{itemize}
\end{frame}

\begin{frame}[fragile]
\frametitle{Choosing a Fabric}
\begin{itemize}[<+(1)->]
\item Cost
\item Durability
\item Washability
\item Colorfastness
\item Convenience
\end{itemize}
\end{frame}

\begin{frame}[fragile]
\frametitle{Clothing Considerations}
\begin{itemize}[<+(1)->]
\item Texture
\item Breathability
\item Drape
\item Opacity
\item Stretch
\end{itemize}
\end{frame}

\subsection{Sewing Tools}

\begin{frame}[fragile]
\frametitle{Marking Fabric}
\begin{itemize}[<+(1)->]
\item Chalk
\item Carbon paper
\item Pencil
\item Markers
\item Sharpie
\end{itemize}
\end{frame}

\begin{frame}[fragile]
\frametitle{Cutting}
\begin{itemize}[<+(1)->]
\item Rotary cutter and mat
\item Scissors
\item Razor blade to remove stitches
\item Quick-unpick
\end{itemize}
\end{frame}

\begin{frame}[fragile]
\frametitle{Holding Stuff}
\begin{itemize}[<+(1)->]
\item Pins
\item Safety pins
\item Clips or clothespins
\item Weights
\end{itemize}
\end{frame}

\begin{frame}[fragile]
\frametitle{Sticking Thread Through Fabric}
\begin{itemize}[<+(1)->]
\item Needle
\item Thimble or sailor's palm
\item Awl
\item Sewing Machine
\end{itemize}
\end{frame}

\begin{frame}[fragile]
\frametitle{Changing fabric's shape}
\begin{itemize}[<+(1)->]
\item Iron
\item Ironing board
\item Hair straightener sometimes works
\item Spray starch
\end{itemize}
\end{frame}

\subsection{Where To Get Materials}

\begin{frame}[fragile]
\frametitle{Fabric store}
\begin{itemize}[<+(1)->]
\item Expensive, even with coupons
\item Fabric comes on bolts
\item Fiber content clearly labeled
\item Priced by the yard
\item WASH BEFORE USING
\end{itemize}
\end{frame}

\begin{frame}[fragile]
\frametitle{Manufacturer}
\begin{itemize}[<+(1)->]
\item Pendleton in Portland
\item Sometimes cheap offcuts, rejects
\item More variety for specialty fabrics
\item WASH BEFORE USING
\end{itemize}
\end{frame}

\begin{frame}[fragile]
\frametitle{Custom Order}
\begin{itemize}[<+(1)->]
\item Custom prints: Spoonflower
\item Custom dying: Etsy
\item Or DIY!
\item WASH BEFORE USING
\end{itemize}
\end{frame}

\begin{frame}[fragile]
\frametitle{Recycling}
\begin{itemize}[<+(1)->]
\item Cheap
\item Don't always know fiber content
\item Fabric comes in funny shapes
\item Unpick seams, iron flat
\item Try sheets, tablecloths, curtains
\item WASH BEFORE USING
\end{itemize}
\end{frame}

\begin{frame}[fragile]
\frametitle{Wash your fabrics}
\begin{itemize}[<+(1)->]
\item Some shrink
\item Some bleed colors
\item Dry clean only is often BS
\end{itemize}
\end{frame}


\begin{frame}[fragile]
\frametitle{Your Homework}
\begin{itemize}[<+(1)->]
\item Read the tags of your own garments
\item Guess fiber content by touch when shopping
\item Guess knit/woven/other by touch in your closet or camp
\end{itemize}
\end{frame}

\section{Attaching Materials Together}

\begin{frame}[fragile]
\frametitle{Attachment Methods}
\begin{itemize}[<+(1)->]
\item Sewing
\item Gluing
\item Melting, fusing
\item Riveting, stapling
\end{itemize}
\end{frame}

\begin{frame}[fragile]
\frametitle{Gluing}
\begin{itemize}[<+(1)->]
\item Easy
\item Prevents fraying
\item Makes a mess
\item Hard to get straight lines
\item Often stiff or crunchy
\item Hard to undo mistakes
\end{itemize}
\end{frame}

\begin{frame}[fragile]
\frametitle{Melting}
\begin{itemize}[<+(1)->]
\item Iron-on tape gives straight line
\item Rarely drapes right
\item Invisible
\item Generally not very strong
\item Prevents fraying
\end{itemize}
\end{frame}

\begin{frame}[fragile]
\frametitle{Riveting}
\begin{itemize}[<+(1)->]
\item Can be stronger, if fabric is reinforced
\item Can tear out
\item Adds metal
\end{itemize}
\end{frame}

\subsection{Sewing}

\begin{frame}[fragile]
\frametitle{Pros and cons}
\begin{itemize}[<+(1)->]
\item Flexible
\item Durable
\item Usually easy to undo
\item We have machines for it
\item Makes lots of holes in fabric
\item Waterproofing seams is fiddly
\end{itemize}
\end{frame}

\begin{frame}[fragile]
\frametitle{How sewing works}
\begin{itemize}[<+(1)->]
\item Threads go through fabric layers
\item Threads are tightened so layers stay together
\item Threads are fastened so they don't fall out
\end{itemize}
\end{frame}

\begin{frame}[fragile]
\frametitle{Threading your needle}
\begin{itemize}[<+(1)->]
\item Take a piece of thread no longer than you can reach
\item Shorter threads are harder to get tangled
\item Stick one end through the eye of the needle
\item Use threading tool to pull it through if necessary
\item Optionally, double thread over so ends are together
\item Knot thread at other end from needle
\item Wax thread with a candle
\end{itemize}
\end{frame}

\begin{frame}[fragile]
\frametitle{Why I wax thread}
\begin{itemize}[<+(1)->]
\item Sticks 2 threads together
\item Flattens microscopic fuzzy bits of thread
\item Reduces thread tangling
\item Sliding through fabric easier reduces thread breakage
\end{itemize}
\end{frame}

\begin{frame}[fragile]
\frametitle{Sewing a seam}
\begin{itemize}[<+(1)->]
\item Decide where seam will go
\item Start at one end
\item Hold fabric layers together with something
\item Put needle from one side to the other
\item Pull thread through
\item ~2mm along the seam, put needle through from the side it came out to the other
\item Repeat ad nauseum
\end{itemize}
\end{frame}

\begin{frame}[fragile]
\frametitle{Back Stitch}
\begin{itemize}[<+(1)->]
\item Always stick the needle in at the side it most recently came out of
\item This time, the place you stick it is "backward" from the direction you were sewing
\item Back stitch locks thread in place and makes stronger seam
\end{itemize}
\end{frame}

\begin{frame}[fragile]
\frametitle{Ending a seam}
\begin{itemize}[<+(1)->]
\item Fold fabric at last stitch and catch a little bit on your needle
\item Take thread where it came out of the fabric, wrap around needle tip 2x
\item Pinch the wrapped thread gently so it doesn't go anywhere
\item Pull the needle forward just like any other stitch
\item Cut thread ~1" from the knot
\end{itemize}
\end{frame}

\begin{frame}[fragile]
\frametitle{Sewing Stretchy Fabrics}
\begin{itemize}[<+(1)->]
\item Thread breaks if it's the shortest part
\item Backstitch often
\item Sew a zig-zag line
\end{itemize}
\end{frame}

\begin{frame}[fragile]
\frametitle{Curved seams}
\begin{itemize}[<+(1)->]
\item Mark both sides
\item Adjust as you sew
\item Stitches should always be on the line
\item Pin it a lot
\end{itemize}
\end{frame}

\begin{frame}[fragile]
\frametitle{Your Homework}
\begin{itemize}[<+(1)->]
\item When you touch something made of cloth, look at its seams
\item What color is the thread?
\item How long are the stitches?
\item Can you find any hand stitching?
\end{itemize}
\end{frame}

\section{Getting a Pattern}

\begin{frame}[fragile]
\frametitle{Store-bought patterns}
\begin{itemize}[<+(1)->]
\item Several sizes on one pattern
\item YouTube can help
\item Look up pattern markings to align grain of fabric
\item Follow directions
\end{itemize}
\end{frame}

\subsection{Make your own pattern}

\begin{frame}[fragile]
\frametitle{Decide what to make}
\begin{itemize}[<+(1)->]
\item Planes are easier than curved surfaces
\item Straight lines are easier to sew than curves
\item Draw the thing you want
\item Or work from an existing thing!
\end{itemize}
\end{frame}


\begin{frame}[fragile]
\frametitle{Identify the planes}
\begin{itemize}[<+(1)->]
\item Measure the planes of the thing
\item Draw them on paper
\item Guess when uncertain
\item Copy papercraft, free patterns for ideas
\end{itemize}
\end{frame}


\begin{frame}[fragile]
\frametitle{Build a paper mockup}
\begin{itemize}[<+(1)->]
\item Cut out the pieces you guessed at
\item Tape them together
\item Note what's wrong
\item Make larger by cutting then taping in more paper
\item Make smaller by folding the paper and taping it up
\end{itemize}
\end{frame}


\begin{frame}[fragile]
\frametitle{Cut up the mockup}
\begin{itemize}[<+(1)->]
\item Draw lines that look like good places for seams
\item Make marks for piecing
\item Cut along the lines, not the marks
\item Transfer pattern pieces to fresh paper if they're a mess
\item Label your pieces!
\end{itemize}
\end{frame}

\begin{frame}[fragile]
\frametitle{Testing Patterns}
\begin{itemize}[<+(1)->]
\item Scale model
\item Mock up in cheaper fabric with same stretch and drape
\item Adjust pattern accordingly
\end{itemize}
\end{frame}

\begin{frame}[fragile]
\frametitle{Laying out a pattern}
\begin{itemize}[<+(1)->]
\item Consider fabric print and matching
\item Consider fabric grain - the bias stretches
\item Within these constraints, play Tetris to avoid waste
\item Place largest pattern pieces first
\item Sew on a small piece if a corner is missing
\end{itemize}
\end{frame}

\begin{frame}[fragile]
\frametitle{Transfer pattern to fabric}
\begin{itemize}[<+(1)->]
\item Attach paper to cloth with pins or weights
\item Trace line of where to sew
\item Trace any marks for matching pieces
\item Add half an inch seam allowance
\item Cut out around the seam allowance
\item Remember, right sides will be together when sewing
\item Fold fabric in half to place symmetrical patterns
\end{itemize}
\end{frame}

\begin{frame}[fragile]
\frametitle{Pin it up}
\begin{itemize}[<+(1)->]
\item Pin cloth pieces together at piecing marks, then along seams
\item RIGHT SIDES IN
\item Plan how to turn a 3D shape
\item Avoid the doughnut trap
\item Think about sewing order
\end{itemize}
\end{frame}

\begin{frame}[fragile]
\frametitle{Plan sewing order}
\begin{itemize}[<+(1)->]
\item Hardest-to-do-later comes first
\item Usually, small stuff onto large stuff, then large stuff together
\item Attach patches, pockets, etc ASAP
\item Sometimes straps, ears, etc go into seams before sewing
\item Everything outside while sewing will be inside final item
\item Leave space to turn it!
\end{itemize}
\end{frame}

\begin{frame}[fragile]
\frametitle{Combining different sized pieces}
\begin{itemize}[<+(1)->]
\item Taper: Cut make larger piece trapezoidal so one end matches smaller
\item Gather: Sew along larger and bunch up fabric on the thread
\item Pleat: Fold over bits of the longer piece till it's the size of the smaller
\item Baseball cap pieces are tapered
\item Tutus are gathered
\item Kilts are pleated
\end{itemize}
\end{frame}

\begin{frame}[fragile]
\frametitle{Cutting Corners}
\begin{itemize}[<+(1)->]
\item Convex curves: Cut out Vs
\item Concave curves: Straight cuts to seam
\item Don't cut through seam
\item Think how seam allowance lies after turning
\end{itemize}
\end{frame}


\begin{frame}[fragile]
\frametitle{Fixing mistakes}
\begin{itemize}[<+(1)->]
\item Mend cuts in wrong places
\item Pull out bad seams
\item Add missing seams
\item Wash out wrong marks
\end{itemize}
\end{frame}


\begin{frame}[fragile]
\frametitle{Your Homework}
\begin{itemize}[<+(1)->]
\item Examine your backpack, tent, and shoes
\item Contemplate how the pieces of fabric fit together to make a 3D shape
\item Sketch the pattern pieces you see
\item Guess what order they were assembled in
\item Extra credit: Try to make a scale model copy!
\end{itemize}
\end{frame}

\section{Mending}

\begin{frame}[fragile]
\frametitle{Broken Seams}
\begin{itemize}[<+(1)->]
\item Find matching thread
\item Sew where seam should be
\item Old holes are often visible
\item Hide knots inside garment
\end{itemize}
\end{frame}

\begin{frame}[fragile]
\frametitle{Darning holes}
\begin{itemize}[<+(1)->]
\item Darning rebuilds fabric by hand.
\item Choose yarn or thick thread
\item Sew long stitches like warp
\item Weave in long stitches perpendicular, like weft
\item Sew haphazardly over it to make more dense
\end{itemize}
\end{frame}

\begin{frame}[fragile]
\frametitle{Patching}
\begin{itemize}[<+(1)->]
\item Patching covers bad fabric with good.
\item Choose material for patch
\item Use iron on patch, or sew on fabric
\item Press edges of patch to side that will go onto item
\item Patch from inside or outside
\item Pin patch in place, sew around edges
\item Sew edges of hole onto patch, too
\end{itemize}
\end{frame}

\section{Tailoring}

\begin{frame}[fragile]
\frametitle{Taking in}
\begin{itemize}[<+(1)->]
\item Makes item smaller
\item Measure how much smaller you want it
\item Turn inside out, sew parallel to existing seam
\end{itemize}
\end{frame}

\begin{frame}[fragile]
\frametitle{Adding darts}
\begin{itemize}[<+(1)->]
\item Darts make an item smaller in just one spot
\item Pin smallest point and ends of dart
\item Sew like a wide triangle from end to center to end
\end{itemize}
\end{frame}

\begin{frame}[fragile]
\frametitle{Letting Out}
\begin{itemize}[<+(1)->]
\item Opposite of taking in
\item Clothes used to have bigger seam allowance
\item Everything is cheap these days
\item Get off my lawn
\end{itemize}
\end{frame}

\begin{frame}[fragile]
\frametitle{Adding gussets}
\begin{itemize}[<+(1)->]
\item Gussets add new fabric to make a garment larger
\item Make a slit, by cutting cloth or opening seam
\item Hold the fabric in the shape you wish it was
\item Patch the resulting hole
\end{itemize}
\end{frame}

\begin{frame}[fragile]
\frametitle{Hemming}
\begin{itemize}[<+(1)->]
\item ``Supposed'' to be hidden
\item Hand sewing: Short stitches on outside, long stitches on inside
\item Fold twice so raw edge is encased in hem
\item Good place for of iron on hemming tape
\end{itemize}
\end{frame}

\begin{frame}[fragile]
\frametitle{Adding patch pockets}
\begin{itemize}[<+(1)->]
\item Cut patch slightly larger than desired pocket
\item Hem the side that will be open
\item Pin it in place
\item Stitch it on
\item If sewing to a lining, try not to sew through outher layers of garment
\end{itemize}
\end{frame}

\begin{frame}[fragile]
\frametitle{Adding pockets in a seam}
\begin{itemize}[<+(1)->]
\item Choose where in the seam the pocket shall go
\item Cut the stitches to open the seam
\item Attach some fabric to each side
\item Make sure seams end up inside the garment where the pocket attaches
\item Remake the seam you removed, but route it around where the pocket goes
\item Cut away excess fabric
\end{itemize}
\end{frame}

\begin{frame}[fragile]
\frametitle{Enlarging small pockets}
\begin{itemize}[<+(1)->]
\item Detach bottom of pocket from garment, if attached
\item Cut open bottom of pocket
\item Attach some fabric along the cut edge to extend the pocket
\item Sew it shut where the bottom of the new, larger pocket goes
\item Nobody cares what sides your seams end up on because it's all inside the garment
\end{itemize}
\end{frame}

\begin{frame}[fragile]
\frametitle{Shrinking waistbands the right way}
\begin{itemize}[<+(1)->]
\item Remove entire waistband from garment
\item Sew darts to make top of garment the right size
\item Replace waistband
\item Remake one side of fastener
\end{itemize}
\end{frame}

\begin{frame}[fragile]
\frametitle{Shrinking waistbands the easy way}
\begin{itemize}[<+(1)->]
\item Cut waistband in a couple places
\item Overlap the cut edges a bit
\item Sew the overlapped bits together
\end{itemize}
\end{frame}

\begin{frame}[fragile]
\frametitle{Adding Elastic}
\begin{itemize}[<+(1)->]
\item Make a channel where the elastic will go
    \begin{itemize}
    \item Cut rectangle of fabric a bit longer and wider than elastic
    \item Press edges under
    \item Pin to garment
    \item Sew along edges, leaving ends open
    \end{itemize}
\item Thread elastic through channel with safety pin
\item Fasten one end of elastic to garment
\item Try it on and figure out best length for elastic
\item Fasten other end of elastic to garment
\end{itemize}
\end{frame}

\begin{frame}[fragile]
\titlepage
\end{frame}

\end{document}
